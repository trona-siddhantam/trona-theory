
\documentclass[12pt]{article}
\usepackage{amsmath,amssymb}
\usepackage{authblk}
\usepackage{geometry}
\usepackage{graphicx}
\usepackage[numbers,sort&compress]{natbib}
\usepackage{titlesec}
\usepackage{abstract}

\geometry{a4paper, margin=1in}

\titleformat{\section}[block]{\large\bfseries}{\thesection.}{1em}{}
\titleformat{\subsection}[block]{\normalsize\bfseries}{\thesubsection.}{1em}{}

\title{\textbf{Trōṇa Siddhāntam (Pluck Theory): A Unified Wave-Resonance Ontology of Reality}}

\author{Trōṇa Tanrangi}
\affil{Independent Researcher\\\texttt{tronatheory@gmail.com}}

\date{\today}

\begin{document}

\maketitle

\begin{abstract}
This paper presents a novel framework — \textit{Trōṇa Siddhāntam (Pluck Theory)} — which posits that all physical phenomena emerge from the interactions of waves within a continuous, wave-permitting substrate called the \textit{Tarangi}. Particles are interpreted as stable resonant knots, forces as phase interactions, gravity as resonance trajectory bending, and time as emergent entropy. The theory proposes reinterpretations of concepts from quantum field theory, general relativity, and the Standard Model while remaining consistent with key experimental observations. 
\end{abstract}

\section{Introduction}
Modern physics is built on distinct but interconnected frameworks: Quantum Field Theory (QFT), General Relativity (GR), and the Standard Model. While these frameworks are highly predictive, they rely on several independent foundational elements — such as particles, fields, forces, spacetime, and time — without unifying them under a single substrate. This paper proposes an alternative ontology: that all observed phenomena are emergent effects of wave interactions within a universal wave-permitting medium — the \textit{Tarangi}.

\section{Foundational Premise}
The core premise of Pluck Theory is that only waves exist fundamentally. Everything — from particles to spacetime — arises from intersecting and resonating wave patterns. The medium, \textit{Tarangi}, allows all wave types to propagate and interact.

\section{Particles and Fields}
\subsection{Resonant Knots}
Particles are viewed as stable interference patterns or “knots” — called \textit{Trōṇas} — that form when waves intersect constructively in frequency, phase, and mode.

\subsection{Fields as Resonance Zones}
What are traditionally called fields are interpreted as statistical zones where certain waveforms recur with higher probability due to the reinforcement of past resonances.

\section{Interactions as Resonance Dynamics}
\subsection{No Fundamental Forces}
Forces emerge from phase relationships between wave structures:
\begin{itemize}
    \item Phase-locking results in binding (e.g., electromagnetism).
    \item Dephasing results in scattering or decay.
\end{itemize}

\subsection{Charge and Quantization}
Charge and quantum numbers correspond to symmetry of coupling between harmonics. Antiparticles are interpreted as phase inversions.

\section{Spacetime and Gravity}
\subsection{Spacetime as Emergent Geometry}
Spacetime is not fundamental; it is the geometry of wave unfolding. Waves don’t move \textit{through} space — they are the structure of space.

\subsection{Gravity as Resonance Curvature}
Gravitational effects emerge from the curvature of wave trajectories through interference zones around massive knots.

\section{Time and Entropy}
Time is understood as the growth of complexity in the wave system. As wave interactions increase in complexity, the resulting entanglement is perceived as the flow of time.

\section{Cosmic Phenomena}
\subsection{Black Holes}
Black holes are extreme zones of self-trapping resonance. Hawking radiation corresponds to the loss of coherence at the boundaries of these traps.

\subsection{Dark Matter}
Dark matter is composed of stable, non-radiative wave knots that affect surrounding wave dynamics but do not couple electromagnetically.

\subsection{Dark Energy}
Dark energy is the result of residual wave tension in the \textit{Tarangi}, contributing to the large-scale expansion of space.

\section{Reinterpretations of Existing Theories}
\begin{itemize}
    \item \textbf{Copenhagen Interpretation}: Collapse becomes resonance reconfiguration.
    \item \textbf{QFT}: Fields are replaced by recurring resonance zones.
    \item \textbf{Mass (Higgs Mechanism)}: Mass is trapped wave energy.
    \item \textbf{General Relativity}: Gravity is not curvature but trajectory bending in resonance zones.
    \item \textbf{E=mc\textsuperscript{2}}: Mass-energy equivalence remains, but mass is interpreted as localized wave energy.
    \item \textbf{Superposition}: Inherent to the internal wave modes of a \textit{Trōṇa}.
    \item \textbf{Entanglement}: Arises from shared phase geometry.
\end{itemize}

\section{Future Work}
Further formalization using mathematical wave mechanics is needed. Experimental predictions, such as deviations from QFT in vacuum polarization or gravitation without spacetime curvature, could be explored.

\section{Conclusion}
\textit{Trōṇa Siddhāntam} reimagines reality as an interplay of waves within a continuous substrate. While speculative, it remains grounded in physical intuition and offers avenues for unification and experimentation.

\bibliographystyle{unsrt}
\bibliography{citations}

\end{document}
